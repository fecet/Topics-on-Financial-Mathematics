% Options for packages loaded elsewhere
\PassOptionsToPackage{unicode}{hyperref}
\PassOptionsToPackage{hyphens}{url}
%
\documentclass[
]{article}
\usepackage{amsmath,amssymb}
\usepackage{lmodern}
\usepackage{iftex}
\ifPDFTeX
  \usepackage[T1]{fontenc}
  \usepackage[utf8]{inputenc}
  \usepackage{textcomp} % provide euro and other symbols
\else % if luatex or xetex
  \usepackage{unicode-math}
  \defaultfontfeatures{Scale=MatchLowercase}
  \defaultfontfeatures[\rmfamily]{Ligatures=TeX,Scale=1}
\fi
% Use upquote if available, for straight quotes in verbatim environments
\IfFileExists{upquote.sty}{\usepackage{upquote}}{}
\IfFileExists{microtype.sty}{% use microtype if available
  \usepackage[]{microtype}
  \UseMicrotypeSet[protrusion]{basicmath} % disable protrusion for tt fonts
}{}
\makeatletter
\@ifundefined{KOMAClassName}{% if non-KOMA class
  \IfFileExists{parskip.sty}{%
    \usepackage{parskip}
  }{% else
    \setlength{\parindent}{0pt}
    \setlength{\parskip}{6pt plus 2pt minus 1pt}}
}{% if KOMA class
  \KOMAoptions{parskip=half}}
\makeatother
\usepackage{xcolor}
\IfFileExists{xurl.sty}{\usepackage{xurl}}{} % add URL line breaks if available
\IfFileExists{bookmark.sty}{\usepackage{bookmark}}{\usepackage{hyperref}}
\hypersetup{
  pdftitle={Final Quiz},
  pdfauthor={Xie zejian},
  hidelinks,
  pdfcreator={LaTeX via pandoc}}
\urlstyle{same} % disable monospaced font for URLs
\usepackage[margin=1in]{geometry}
\usepackage{longtable,booktabs,array}
\usepackage{calc} % for calculating minipage widths
% Correct order of tables after \paragraph or \subparagraph
\usepackage{etoolbox}
\makeatletter
\patchcmd\longtable{\par}{\if@noskipsec\mbox{}\fi\par}{}{}
\makeatother
% Allow footnotes in longtable head/foot
\IfFileExists{footnotehyper.sty}{\usepackage{footnotehyper}}{\usepackage{footnote}}
\makesavenoteenv{longtable}
\usepackage{graphicx}
\makeatletter
\def\maxwidth{\ifdim\Gin@nat@width>\linewidth\linewidth\else\Gin@nat@width\fi}
\def\maxheight{\ifdim\Gin@nat@height>\textheight\textheight\else\Gin@nat@height\fi}
\makeatother
% Scale images if necessary, so that they will not overflow the page
% margins by default, and it is still possible to overwrite the defaults
% using explicit options in \includegraphics[width, height, ...]{}
\setkeys{Gin}{width=\maxwidth,height=\maxheight,keepaspectratio}
% Set default figure placement to htbp
\makeatletter
\def\fps@figure{htbp}
\makeatother
\setlength{\emergencystretch}{3em} % prevent overfull lines
\providecommand{\tightlist}{%
  \setlength{\itemsep}{0pt}\setlength{\parskip}{0pt}}
\setcounter{secnumdepth}{-\maxdimen} % remove section numbering
\newlength{\cslhangindent}
\setlength{\cslhangindent}{1.5em}
\newlength{\csllabelwidth}
\setlength{\csllabelwidth}{3em}
\newlength{\cslentryspacingunit} % times entry-spacing
\setlength{\cslentryspacingunit}{\parskip}
\newenvironment{CSLReferences}[2] % #1 hanging-ident, #2 entry spacing
 {% don't indent paragraphs
  \setlength{\parindent}{0pt}
  % turn on hanging indent if param 1 is 1
  \ifodd #1
  \let\oldpar\par
  \def\par{\hangindent=\cslhangindent\oldpar}
  \fi
  % set entry spacing
  \setlength{\parskip}{#2\cslentryspacingunit}
 }%
 {}
\usepackage{calc}
\newcommand{\CSLBlock}[1]{#1\hfill\break}
\newcommand{\CSLLeftMargin}[1]{\parbox[t]{\csllabelwidth}{#1}}
\newcommand{\CSLRightInline}[1]{\parbox[t]{\linewidth - \csllabelwidth}{#1}\break}
\newcommand{\CSLIndent}[1]{\hspace{\cslhangindent}#1}
%\setmathfont{texgyrepagella-math.otf}
%\setmathfont{XITSMath-Regular.otf}

\usepackage{booktabs}
\usepackage{graphicx}
\usepackage{subfigure}
\usepackage{amssymb}
\usepackage{mathtools}
\usepackage{upgreek}
\usepackage{float}
\usepackage{extarrows}
\usepackage{longtable}
\usepackage{tikz}
\usepackage{commutative-diagrams}
\usepackage{authblk}
\usepackage{letltxmacro}

\DeclareMathOperator{\tr}{Tr}
\DeclareMathOperator{\rank}{rank}
\DeclareMathOperator{\deter}{det}
\DeclareMathOperator{\diag}{diag}
\DeclareMathOperator{\eig}{eig}
\DeclareMathOperator{\vect}{vec}

\let\emptyset\varnothing

\renewcommand{\mathbf}[1]{\symbf{#1}}
\newcommand{\bm}[1]{\symbf{#1}}


% \setcounter{section}{+7}
\author{Xie Zejian\\ 11810105@mail.sustech.edu.cn}
  \affil{Department of Finance, SUSTech
      %\\ \href{https://www.cbs.nl}{\texttt{https://www.cbs.nl}} \\
        }



\newcommand{\lt}{<}
\newcommand{\gt}{>}
\newcommand{\R}{\mathbb{R}}
\newcommand{\Reals}{\mathbb{R}}
\newcommand{\Real}{\mathbb{R}}
\newcommand{\N}{\mathbb{N}}
\newcommand{\Q}{\mathbb{Q}}
\newcommand{\sub}{\subset}
\newcommand{\subsets}{\subset}
\newcommand{\exist}{\exists}
\DeclareMathOperator*{\argmax}{arg\,max}
\DeclareMathOperator*{\argmin}{arg\,min}
\ifLuaTeX
  \usepackage{selnolig}  % disable illegal ligatures
\fi

\title{Final Quiz}
\author{Xie zejian}
\date{Last compiled on 17:49, 09 January, 2022}

\usepackage{amsthm}
\newtheorem{theorem}{Theorem}
\newtheorem{lemma}{Lemma}
\newtheorem{corollary}{Corollary}
\newtheorem{proposition}{Proposition}
\newtheorem{conjecture}{Conjecture}
\theoremstyle{definition}
\newtheorem{definition}{Definition}
\theoremstyle{definition}
\newtheorem{example}{Example}
\theoremstyle{definition}
\newtheorem{exercise}{Exercise}
\theoremstyle{definition}
\newtheorem{hypothesis}{Hypothesis}
\theoremstyle{remark}
\newtheorem*{remark}{Remark}
\newtheorem*{solution}{Solution}
\begin{document}
\maketitle

\begin{lemma}[]
\protect\hypertarget{lem:mart}{}\label{lem:mart}Process
\[
S_{t}=S_0 \exp \left\{ \int_{0}^{t} \sigma_{s}dB_{s}-\int_{0}^{t} \frac{1}{2}\sigma^2_{s} ds  \right\}
\]
is a martingale.
\end{lemma}

\begin{proof}
Let \(X_{t}=\int_{0}^{t} \sigma_{s}dB_{s}-\int_{0}^{t} \frac{1}{2}\sigma^2_{s} ds\), we have
\[
\begin{cases}
    dX_{t}&= \sigma_{t}dB_{t}-\frac{1}{2}\sigma^2_{t}dt \\
    dX_{t}dX_{t}&=\sigma^2_{t}dB_{t}dB_{t}=\sigma^2_{t}dt
\end{cases}
\]
Let \(f(x)=S_0 e^{x}\), by Ito-Doeblin formula:
\[
\begin{aligned}
   dS_{t}&=f'(X_{t})dX_{t}+\frac{1}{2}f''(X_{t})dX_{t}dX_{t}
   \\ &= 
   S_0e^{X_{t}}dX_{t}+\frac{1}{2}S_0 e^{X_{t}}dX_{t}dX_{t}
   \\ &= 
   \sigma_{t}S_{t}dB_{t}
\end{aligned}
\]
thus \(S_{t}\) is martingale by theorem II.20 (Protter 2005).
\end{proof}

\begin{lemma}[]
\protect\hypertarget{lem:normal}{}\label{lem:normal}For deterministic function \(\Delta(s)\),
\[
I(t):= \int _{0}^{t} \Delta(s)d B_{s} \sim \mathcal{N}\left( 0, \int _{0}^{t} \Delta^2(s)ds \right)
\]
\end{lemma}

\begin{proof}
As \(I(t)\) is martingale, we have \(\mathop{{}\mathbb{E}}_{}I(t)=\mathop{{}\mathbb{E}}_{}I(0)=0\), and by Ito's isometry:
\[
\mathop{\text{Var}}I(t)=\mathop{{}\mathbb{E}}_{}I^2(t)=\int_{0}^{t} \Delta^2(s)ds 
\]
There is remain to show it's normally distributed, \(\text{ i.e. }\),
\[
\mathop{{}\mathbb{E}}_{}e^{uI(t)}= \exp \left\{  \frac{1}{2} u^2 \int_{0}^{t} \Delta^2(s)ds   \right\}
\]
that is
\[
\mathop{{}\mathbb{E}}_{}\exp \left\{ \int_{0}^{t} u \Delta(s)dB_{s}-  \frac{1}{2}  \int_{0}^{t} \left[ u\Delta(s)\right]^2ds   \right\}=1
\]
by lemma \ref{lem:mart},

\[
\exp \left\{ \int_{0}^{t} u \Delta(s)dB_{s}-  \frac{1}{2}  \int_{0}^{t} \left[ u\Delta(s)\right]^2ds   \right\}
\]
is martingale, and it's start with \(1\) clearly, this completes the proof.
\end{proof}

\begin{exercise}[]
\(\quad\)
\end{exercise}

\begin{solution}
By Ito-Doeblin formula in integral form, take \(f=x \mapsto \frac{1}{2}x^2\):
\[
\begin{aligned}
    \frac{1}{2}B_{t}^2=f(B_{t})-f(B_0)&=\int_{0}^{t} f'(B_{s})dB_{s}+\frac{1}{2} \int _{0}^{t} f''(B_{s})ds
    \\ &=  \int_{0}^{t} B_{s} dB_{s}+\frac{t}{2}
\end{aligned}
\]
thus
\[
\int_{0}^{t} B_{s}dB_{s}=\frac{1}{2}B_{t}^2-\frac{t}{2}
\]
\end{solution}

\begin{exercise}[]
\(\quad\)
\end{exercise}

\begin{solution}
By theorem II.39(Protter 2005), \(X_{t}\) is a Brownian motion.
\[
\begin{aligned}
    \mathop{{}\mathbb{E}}_{s}\exp \left\{ iuX_{t}+ \frac{u^2t}{2} \right\}
    &=
    \mathop{{}\mathbb{E}}_{s} \left[ \exp \left\{ iu(X_{t}-X_{s}) \right\} \cdot \exp \left\{ iuX_{s}+\frac{u^2t}{2} \right\} \right]
    \\ &= 
    \exp \left\{ iuX_{s}+\frac{u^2t}{2} \right\}\cdot\mathop{{}\mathbb{E}}_{s} \left[ \exp \left\{ iu(X_{t}-X_{s}) \right\} \right]
    \\ &= 
    \exp \left\{ iuX_{s}+\frac{u^2t}{2} \right\}\cdot\mathop{{}\mathbb{E}}_{} \left[ \exp \left\{ iu(X_{t}-X_{s}) \right\} \right] \text{(Stationary increments)}
    \\ &= 
    \exp \left\{ iuX_{s}+\frac{u^2t}{2} \right\}\cdot\mathop{{}\mathbb{E}}_{} \left[ \exp \left\{ iu \mathfrak{Z} \sqrt{t-s} \right\} \right]
    \\ &= 
    \exp \left\{ iuX_{s}+\frac{u^2t}{2} \right\} \exp \left\{ -\frac{u^2(t-s)}{2} \right\} \text{(MGF of normal distribution)}
    \\ &= 
    \exp \left\{ iuX_{s}+\frac{u^2s}{2} \right\}     
\end{aligned}
\]
The integrability follows from \(X_{t}\) is Brownian motion:
\[
\mathop{{}\mathbb{E}}_{}\exp \left\{ iuX_{t}+ \frac{u^2t}{2} \right\}= \frac{u^2t}{2} \mathop{{}\mathbb{E}}_{}\exp \left\{ iuX_{t} \right\}=\frac{u^2t-u^2t}{2}=0
\]
this completes the proof.
\end{solution}

\begin{exercise}[]
\(\quad\)
\end{exercise}

\begin{solution}
Let \(Y_{t}=X_0+\sigma \int_{0}^{t} e^{\alpha s} d B_{s}\), we have
\[
\begin{cases}
    dY_{t} &= \sigma e^{\alpha t}dB_{t} \\
    dY_{t}dY_{t} &= \sigma^2 e^{2\alpha t}dt \\
\end{cases}
\]
Let \(X_{t}=f(t,Y_{t})=e^{-\alpha t}Y_{t}\), \(f_{t}(t,Y_{t})=- \alpha e^{-\alpha t}Y_{t},f_{x}=e^{-\alpha t},f_{xx}=0\), Ito-Doeblin formula yields,
\[
\begin{aligned}
    dX_{t}&=-\alpha  e^{-\alpha t}  Y_{t}dt+e^{-\alpha t}dY_{t}
    \\ &= 
    -\alpha X_{t}dt+\sigma dB_{t}
\end{aligned}
\]
\end{solution}

\begin{exercise}[]
\(\quad\)
\end{exercise}

\begin{solution}
By lemma \ref{lem:normal}, take \(\Delta(s)=s\), we have
\[
\int_{0}^{t} s^2ds=\frac{1}{3}
\]
then the claim follows.
\end{solution}

\begin{exercise}[]
\(\quad\)
\end{exercise}

\begin{solution}

\begin{enumerate}
\def\labelenumi{\alph{enumi}.}
\item
  Let \(f(t,x)=e^{\beta t}x\), by Ito-Doeblin formula:
  \[
  \begin{aligned}
   d (e^{\beta t}R_{t}) &= df(t,R_{t}) = \beta e^{\beta t} R_{t}dt+e^{\beta t} 
   \left[ \left( \alpha - \beta R_{t} \right)dt+\sigma \sqrt{R_{t}}dB_{t} \right]
   \\ &= 
   \alpha e^{\beta t}dt+\sigma e^{\beta t}\sqrt{R_{t}}dB_{t}
  \end{aligned}
  \]
  Integrate each sides:
  \[
  \begin{aligned}
   e^{\beta t}R_{t} &= R_0+ \frac{\alpha}{\beta} (e^{\beta t}-1)+\sigma \int_{0}^{t}  e^{\beta s}\sqrt{R_{s}} dB_{s} 
  \end{aligned}
  \]
  As Integration \(\text{ w.r.t } B_{t}\) is martingale and thus have zero expectation:
  \[
  e^{\beta t} \mathop{{}\mathbb{E}}_{}R_{t}=R_0+\frac{\alpha}{\beta} (e^{\beta t}-1)
  \]
  that is
  \[
  \mathop{{}\mathbb{E}}_{}R_{t}=e^{-\beta t} r_0+ \frac{\alpha(1-e^{-\beta t})}{\beta}
  \]
\item
  Let \(X_{t}=e^{\beta t}R_{t}\), we already have
  \[
  dX_{t}=
  \alpha e^{\beta t}dt+\sigma e^{\beta t}\sqrt{R_{t}}dB_{t}
  =
  \alpha e^{\beta t}dt+\sigma e^{\frac{\beta t}{2}}\sqrt{X_{t}}dB_{t}
  \]
  and
  \[
  \mathop{{}\mathbb{E}}_{}X_{t}=r_0+\frac{\alpha}{\beta} (e^{\beta t}-1)
  \]
  Let \(f(X)=x^2\), Ito-Doeblin formula yields:
  \[
  d(X^2(t))=2\alpha e^{\beta t} X_{t} dt+2\sigma e^{\frac{\beta t}{2}}X^{\frac{3}{2}}_{t}dB_{t}+\sigma^2e^{\beta t}X_{t}dt
  \]
  Integrate each sides, we have
  \[
  X^2_{t}=X^2_{0}+(2\alpha+\sigma^2)\int_{0}^{t} e^{\beta s}X_{s}ds+2\sigma \int_{0}^{t} e^{\frac{\beta s}{2}}X^{\frac{3}{2}}_{s}dW_{s}
  \]
  Take expectation each sides:
  \[
  \begin{aligned}
   \mathop{{}\mathbb{E}}_{}X^2 &= X^2_{0}+(2\alpha +\sigma^2)\int_{0}^{t} e^{\beta s}\mathop{{}\mathbb{E}}_{}X_{s}ds \text{ (Fubini's theorem)}
   \\ &= 
   r_0^2+(2\alpha+\sigma^2 ) \int_{0}^{t} e^{\beta s} \left[ r_0+\frac{\alpha(e^{\beta s}-1)}{\beta} \right]ds 
   
  \end{aligned}
  \]
  Then we can derivate \(\mathop{\text{Var}}R_{t}\) by
  \[
  \begin{aligned}
   \mathop{\text{Var}}R_{t}&=\mathop{{}\mathbb{E}}_{}R^2_{t}-(\mathop{{}\mathbb{E}}_{}R_{t})^2=e^{-2\beta t} \mathop{{}\mathbb{E}}_{}X^2_{t}
   
   
   \\ &= 
   \frac{\sigma^2r_0}{\beta}(e^{-\beta t}-e^{-2\beta t})+\frac{\alpha \sigma^2}{2\beta^2} (1-2e^{-\beta t}+e^{-2\beta t})
  \end{aligned}
  \]
\end{enumerate}

\end{solution}

\hypertarget{reference}{%
\section*{Reference}\label{reference}}
\addcontentsline{toc}{section}{Reference}

\hypertarget{refs}{}
\begin{CSLReferences}{1}{0}
\leavevmode\vadjust pre{\hypertarget{ref-protter2005stochastic}{}}%
Protter, Philip E. 2005. \emph{Stochastic Differential Equations}. Springer.

\end{CSLReferences}

\end{document}

\begin{figure}[htpb]
    \centering
    \includegraphics[width=0.8\textwidth]{}
    \caption{}
    \label{fig:}
    f
\end{figure}

